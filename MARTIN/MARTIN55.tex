\documentclass[10pt,A4paper]{article}
\usepackage{zed-csp,graphicx,color}%from
\begin{document}
\pagenumbering{roman}
\begin{titlepage}
 \begin{figure}[h]
  \centerline{\small MAKERERE 
  \includegraphics[width=0.1\textwidth]{muk_log} UNIVERSITY}
\end{figure}
\centerline{COLLEGE OF COMPUTING AND INFORMATIC SCIENCES}
\paragraph{•}
\centerline{DEPARTMENT OF COMPUTER SCIENCE\\}
\paragraph{•}

\centerline{COURSEWORK: RESEARCH METHODOLOGY(BIT 2207)\\}
\paragraph{•}

\centerline{LECTURER: MR.ERNEST MWEBAZE}
\paragraph{•}

\centerline{TOPIC\\}SOLUTION OF HOW TO SURVIVE IN LUMUMBA HALL AMIDST VIOLENCE \\
\paragraph{•}
\centerline{COMPILED BY: \
 KITIYO MARTIN}
 \paragraph{•}
\centerline{STUDENT NUMBER : 216020769}
\paragraph{•}
\centerline{REGISTRATION NUMBER:16/U/18802}
\paragraph{•}
\begin{flushright}
    Signature ....................\\
    DATE: $ FEBRUARY,7{TH},2018$
\end{flushright}
\end{titlepage}
\newpage
\tableofcontents
\newpage
\pagenumbering{arabic}
\section{Introduction}
The great lumumba is one of the hall of residence in makerere university(Uganda),its believed that to survive in lumumba is a matter of the strongest.
To survive means to live peacefully amongst  too many distractions caused by various unwanted objects
\section{PROBLEM STATEMENT}
Due the believe that lumumba is the most violent hall of residence in makerere university causes many students to fear even associating with those resident of lumumba ,these to cause the hall to receive very few student who are attached there to be resident many of them choosing to reside outside the campus in hostels.
\section{ PURPOSE OF THE STUDY}
\subsection{Major objective}
To change students attitude towards lumumba.
\subsection{specific objective}
To create unity among the halls in makerere.\\
To create unity among the students\\
To reduce the level of violence in lumumba.
\section{SIGNIFICANCE OF THE STUDY}
The main reason of the study is to develop positive change in lumumba.
\section{RESEARCH QUESTIONS}
The questions asked includes the following:How do you feel about lumumba?\\How do lumumbist behave?\\Would you live in lumumba given a chance?.
\section{SCOPE OF THE RESEARCH}
The research was carried out in lumumba hall of residence,resident,non resident  including those students victimised by lumumbist were taken  int consideration.
\section{Results}
 Through various  methods like interviewing, sampling, questionnaire,observation i was able discover the followings techniques are applied by various lumumbist to survive include:
 \subsubsection{making friends}
 Those with very many friends tend to survive better because as saying goes"unity is strength but divided we fall".
 \subsubsection{Be obedient}
 The easiest way to survive is to be humble and to obey the halls rules and regulation one of the resident students told me they have weird culture but you obey to live peaceful.
 \subsubsection{spraying of rooms} 
 To eradicate the dangerous pest(bedbugs) its advisable to spray the Rooms to reduce on their attack and peacefully.
\section{Analysis}
 

\section{Conclusion}
From the results it can be seen that the only to survive is through socialise with fellow students and be obedient to the rules abiding them together.
\section{Suggestions for Further Research}
The causes of the various strikes in lumumba hall need to be researched upon in coming  future.
\section{Types of research  Methods  used}
I was able to apply various research methods including the following:
applied research was used because i need to solve the problem of many students suffering in lumumba.
Analytical research was used because am using an already established facts about those students who have enjoyed staying in lunumba and those victimised by lumumbist.
Quantitative research since the number of students interviewed was  expressed in numerical  form.
\section{reference}
GRC LUMUMBA HALL,
CHAIRMAN LUMUMBA HALL.
\end{document}
